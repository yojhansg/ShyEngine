%---------------------------------------------------------------------
%
%                          Parte 3
%
%---------------------------------------------------------------------
%
% Parte3.tex
% Copyright 2009 Marco Antonio Gomez-Martin, Pedro Pablo Gomez-Martin
%
% This file belongs to the TeXiS manual, a LaTeX template for writting
% Thesis and other documents. The complete last TeXiS package can
% be obtained from http://gaia.fdi.ucm.es/projects/texis/
%
% Although the TeXiS template itself is distributed under the 
% conditions of the LaTeX Project Public License
% (http://www.latex-project.org/lppl.txt), the manual content
% uses the CC-BY-SA license that stays that you are free:
%
%    - to share & to copy, distribute and transmit the work
%    - to remix and to adapt the work
%
% under the following conditions:
%
%    - Attribution: you must attribute the work in the manner
%      specified by the author or licensor (but not in any way that
%      suggests that they endorse you or your use of the work).
%    - Share Alike: if you alter, transform, or build upon this
%      work, you may distribute the resulting work only under the
%      same, similar or a compatible license.
%
% The complete license is available in
% http://creativecommons.org/licenses/by-sa/3.0/legalcode
%
%---------------------------------------------------------------------

% Definici�n de la �ltima parte del manual, los ap�ndices

\partTitle{Ap�ndices}

%---------------------------------------------------------------------
%
%                          Resumen EN
%
%---------------------------------------------------------------------

\section{Abstract}
\cabeceraEspecial{Abstract}

The development of video games is a field that has undergone significant evolution over the years. Initially, video games were simple
programs with basic behavior and very limited graphics, which generally provided little entertainment value. Over time, and partly due
to technological advancements, they have considerably increased in complexity and scope. Nowadays, video game development involves the 
creation of interactive virtual worlds that can encompass a wide variety of genres and platforms, from mobile games to high-end titles
for consoles and computers.

\medskip

Hence, the most commonly used and powerful tools for game development today require a notable knowledge base in development and have a
steep learning curve. Therefore, they pose a barrier for those individuals who want to get started in the world of development or who
simply want to create their first simple video game without the necessary knowledge or experience.

\medskip

With this project, we aim to simplify the game development process for individuals inexperienced in programming or development in general.
This involves in-depth study of what a video game is, how they are created, the various 2D game engines available and their architectures,
as well as the different ways to program the behavior of game elements. All of this is done with the goal of designing a game editor and 
engine that make game creation accessible and easy to understand for those who lack experience in the field.



%---------------------------------------------------------------------
%
%                          Introduccion EN
%
%---------------------------------------------------------------------

%-------------------------------------------------------------------
\section{Motivation}
\label{cap2:sec:motivation}

Video game programming has historically been an extremely complex and laborious process. In its early days, developers faced the challenge of creating 
games from scratch, building each element and functionality manually. In addition, they relied heavily on external libraries to access essential resources.
This approach, while rewarding in terms of control and customization, significantly limited accessibility to the world of game creation.    

\medskip

In contrast, today we have witnessed a revolution in the game development industry. The creation of video games has become considerably more accessible 
thanks to the emergence of powerful editors and development engines. These editors offer a range of tools and resources that greatly simplify the creation
process, allowing developers to concentrate on creativity and gameplay rather than worrying about overwhelming technical details.

\medskip

Despite the accessibility provided by modern editors, the learning curve can be steep, especially for those who are new to the world of programming and
game development. This project focuses on creating an engine and editor that allow for the development of simple games while also having a more accessible
learning curve for individuals without experience in this field.

%-------------------------------------------------------------------
\section{Goals}
\label{cap2:sec:goals}

The main goal is to develop a self-contained 2D game engine with an integrated editor and visual node-based programming. Additionally, it will be possible
to test and view the progress of the game being developed directly from the editor and generate executable versions.

\medskip

With this, users will have a tool to develop any type of 2D game with an accessible level of complexity and a pleasant and intuitive user experience.

\section{Tools}
\label{cap2:sec:tools}

%-------------------------------------------------------------------
To start, Git has been used as a version control system through the GitHub Desktop application. All implemented code has been uploaded to a repository,
with the work divided into branches.

\medskip

Link to the repository: \url{https://github.com/ivasan07/ShyEngine}

\medskip

The code has been developed in the Visual Studio 2022 integrated development environment (IDE) and written in C++.

\medskip

Lastly, we have managed tasks using the Trello project management system.

%-------------------------------------------------------------------
\section{Work Plan}
\label{cap2:sec:workplan}

Our project will be divided into three main blocks: engine, editor and visual scripting.

\medskip

Additionally, the work will be divided into five phases: research and planning, initial development, integration of the projects and developemente core,
improvements and development closure, and user testing.

\begin{enumerate}
    \item \textit{Research and planning}: The first phase of the work will involve researching various game engines and editors to understand their 
    functionality and different architectures. We will also look for libraries that fit the demands of our project to achieve comfortable and efficient
    development. Finally, we will plan the division of tasks and the future integration of each part.

    \item \textit{Initial development}: For this phase, we will develop the core of each project. Concerning the engine, we will create initial projects
    to test the libraries to be used and implement the basic game architecture. Regarding the editor, we will create the project in which the selected
    graphical interface library will be integrated to test its functionality. As for visual scripting, we will begin prototyping the language.

    \item \textit{Integration and developement core}: Once the core of each project is developed, we will integrate them in a way that we have a functional
    flow between the three. In other words, we will implement a straightforward development flow in which logic is created from the editor and can be executed
    in the engine.

    \item \textit{Improvements and development closure}: With the three projects integrated, we will continue the development of common functionalities 
    to make it as polished as possible before user testing. This will involve testing/developing a game to identify any potential errors and correct them.

    \item \textit{User testing}: User testing will be conducted with individuals from different backgrounds: users with programming experience and users 
    without experience. This way, we will leverage their feedback to fix potential errors and refine details that enhance the user experience.
\end{enumerate}



%---------------------------------------------------------------------
%
%                          Conclusiones EN
%
%---------------------------------------------------------------------

The aim of this final degree project was to develop a 2D game engine for non-programmers. We have successfully
achieved the proposed goal. Our engine opens the doors to developers with little programming experience while
providing enough functionality to create competent 2D video games.

In practical applications, our engine can be used in the indie game development world or even for educational purposes.

From a technical perspective, we have reached the following conclusions:

\begin{itemize}

    \item With the current implementation, ImGUI windows cannot be moved outside the main SDL window, which can be 
    inconvenient in some situations, such as when implementing a script where it's essential to view the scene or some 
    editor parameters to make decisions.

    \item We had to implement reflection in C++. It would have been more convenient to choose a language with reflection, 
    which would have provided us with greater flexibility.

\end{itemize}

As for future work, we are considering the following functionalities:

\begin{itemize}

    \item Enabling the game to be launched within the editor itself rather than in a separate window.

    \item Modifying the current editor implementation to allow the movement of ImGUI windows outside the main SDL window.

    \item Adding an animation and rendering system to reduce the dependence on external tools.

    \item Abstracting the user from file paths and directories by converting files into engine objects.

    \item To delve deeper into scripting, adding more complex functionality that allows the user greater expressiveness, 
    for example, by adding editable arrays from the editor, class creation, recursion, timers, coroutines, and debugging 
    of running nodes.

\end{itemize}


\makepart
