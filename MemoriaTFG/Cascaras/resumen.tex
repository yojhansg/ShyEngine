%---------------------------------------------------------------------
%
%                      resumen.tex
%
%---------------------------------------------------------------------
%
% Contiene el cap�tulo del resumen.
%
% Se crea como un cap�tulo sin numeraci�n.
%
%---------------------------------------------------------------------

\chapter{Resumen}
\cabeceraEspecial{Resumen}

% \begin{FraseCelebre}
% \begin{Frase}
% Si hubiera tenido m�s tiempo, habr�a escrito una carta m�s corta.
% \end{Frase}
% \begin{Fuente}
% Blaise Pascal
% \end{Fuente}
% \end{FraseCelebre}


El desarrollo de videojuegos es un campo que ha experimentado una evoluci�n significativa a lo largo de los a�os. Inicialmente, los videojuegos eran programas 
simples con un comportamiento b�sico y con gr�ficos muy reducidos pero con el tiempo, y debido en parte a la evoluci�n de la tecnolog�a, han aumentado 
considerablemente en complejidad y alcance. 

\medskip

Un motor de ejecuci�n es un software que proporciona la teconolog�a necesaria para desarrollar el gameplay de un videojuego. Es un conjunto de librer�as 
(gr�ficos, audio, f�sicas...) agrupadas de forma coherente que abstraen la teconolog�a al desarrollador para que pueda centrarse en el desarrollo del videojuego.

\medskip

Hoy en d�a, el desarrollo de videojuegos es una tarea compleja. Por un lado, se puede llevar a cabo desde cero, es decir, programando la teconolog�a que 
requerir� el videojuego y posteriormente desarroll�ndolo y, por otro lado, usando un motor de ejecuci�n que proporcione esa tecnolog�a independiente a las
necesidades concretas del videojuego a desarrollar.

\medskip

En algunos casos, con el fin de ayudar al desarrollo, se incorporan los editores de videojuegos. Los editores simplifican el desarrollo al unir la creaci�n de
elementos de juego, definici�n de comportamiento, depuraci�n y generaci�n de versiones ejecutables, entre otras cosas, en una sola herramienta visual. Esto 
evita tener que comunicarse directamente con el motor de ejecuci�n a trav�s de la programaci�n. 

\medskip

Esto supone una gran ventaja a los desarrolladores experimentados pero motores como Unity o Unreal Engine pueden albergar demasiada complejidad para personas
sin experiencia en programaci�n, incluso aunque su objetivo sean juegos sencillos en 2D. Adem�s, son sistemas enormes que al pretender servir para hacer 
cualquier juego tienen mucha funcionalidad variada y crean versiones ejecutables con gran cantidad de datos innecesarios. Si se quieren hacer juegos peque�os 
y sencillos, el peaje que se paga es muy grande.

\medskip

Aqu� es donde entra en juego nuestro trabajo. La idea es hacer un motor con su editor para hacer juegos peque�os en los que la experiencia de desarrollo sea
equivalente a la de los editores de motores m�s grandes, pero que est� centrado en el desarrollo de juegos 2D m�s peque�os y suponga una carga mucho menor
en ejecuci�n y a la hora de generar las versiones ejecutables. Esto abrir� las puertas a nuestro motor a desarrolladores con poca experiencia en programaci�n
o incluso perfiles sin experiencia ninguna en desarrollo de videojuegos.

\endinput

% Variable local para emacs, para que encuentre el fichero maestro de
% compilaci�n y funcionen mejor algunas teclas r�pidas de AucTeX
%%%
%%% Local Variables:
%%% mode: latex
%%% TeX-master: "../Tesis.tex"
%%% End:
