%---------------------------------------------------------------------
%
%                      resumen.tex
%
%---------------------------------------------------------------------
%
% Contiene el cap�tulo del resumen.
%
% Se crea como un cap�tulo sin numeraci�n.
%
%---------------------------------------------------------------------

\chapter{Resumen}
\cabeceraEspecial{Resumen}

% \begin{FraseCelebre}
% \begin{Frase}
% Si hubiera tenido m�s tiempo, habr�a escrito una carta m�s corta.
% \end{Frase}
% \begin{Fuente}
% Blaise Pascal
% \end{Fuente}
% \end{FraseCelebre}

\section*{Editor y motor de juegos 2D para no programadores}

Un motor de videojuegos es un entorno de desarrollo que proporciona herramientas para
la creaci�n de videojuegos. Estas herramientas evitan al desarrollador
implementar gran cantidad de funcionalidad para centrase en mayor medida 
en el desarrollo del videojuego. Algunos ejemplos de funcionalidades que aportan
los motores son: rendereizado gr�fico, motor de f�sicas, sistema de audio, gesti�n de input 
del jugador, gesti�n de recursos, gesti�n de red, etc.
 
Adem�s, pueden llevar integrado un editor. Los editores son herramientas
visuales cuyo objetivo es comunicar al motor las acciones que realiza el desarrollador. 
Por lo tanto, forman parte del entorno de desarrollo del motor. Los editores suelen tener una 
curva de aprendizaje, especialmente para aquellos que no est�n familiarizados con el motor en 
particular o con el desarrollo de videojuegos en general. Sin embargo, una vez que los desarrolladores
se familiarizan con las herramientas, pueden acelerar significativamente el proceso de creaci�n del 
juego y mejorar la productividad.

Esto supone una gran ventaja a los desarrolladores experimentados pero motores como
Unity o UnrealEngine pueden albergar demasiada complejidad para personas sin experiencia
en programaci�n, incluso aunque su objetivo sean juegos sencillos en 2D. Una herramienta 
muy �til para solucionar este problema es la programaci�n visual. Este tipo de programaci�n 
permite a los usuarios crear l�gica mediante la manipulaci�n de elementos gr�ficos en lugar 
de especificarlos exclusivamente de manera textual. Unity cuenta con su Unity Visual Scripting
y UnrealEngine con los Blueprints.

El motor de este trabajo de fin de grado consiste en un entorno de desarrollo 
de videojuegos 2D autosuficiente. Esto quiere decir que permitir� gestionar los recursos 
del videojuego, las escenas y los elementos interactivos. Adem�s, dar� soporte para la 
creaci�n de  comportamientos a trav�s de programaci�n visual basada en nodos, la ejecuci�n del 
juego en el editor y la creaci�n de ejecutables finales del juego para su distribuci�n.

\endinput
% Variable local para emacs, para  que encuentre el fichero maestro de
% compilaci�n y funcionen mejor algunas teclas r�pidas de AucTeX
%%%
%%% Local Variables:
%%% mode: latex
%%% TeX-master: "../Tesis.tex"
%%% End:
