%---------------------------------------------------------------------
%
%                          Estado del arte
%
%---------------------------------------------------------------------

\chapter{Estado del arte}\label{chap:estado_del_arte}

% \begin{resumen}
    % En este cap�tulo se explica qu� son los motores de video-
    % juegos 2D y cu�les hemos tomado como referencia para desarrollar el
    % nuestro. Asimismo, se expondr�n las librer�as que hemos decidido emplear en nuestro proyecto y se 
    % justificar� la elecci�n de cada una. Adem�s, se menciona la diferencia entre el desarrollo de un
    % videojuego a trav�s de scripting y programaci�n. Por �ltimo, se habla sobre
    % el papel que juegan los motores en el mercado actual de los
    % videojuegos.
% \end{resumen}

%-------------------------------------------------------------------
\section{Motores de videojuegos 2D}
%-------------------------------------------------------------------
\label{cap2:sec:motores}

Explicar qu� es un motor de videojuegos 2D.

%-------------------------------------------------------------------
\section{Referencias de motores de videojuegos 2D}
%-------------------------------------------------------------------
\label{cap2:sec:referencias}

Hablar sobre los motores que hemos tomado como referencia.

%-------------------------------------------------------------------
\section{Editores en motores de videojuegos}
%-------------------------------------------------------------------
\label{cap2:sec:editores}

Explicar qu� es un editor en el �mbito de motores de videojuegos y qu� importancia tienen para el usuario.

%-------------------------------------------------------------------
\section{Scripting vs programaci�n}
%-------------------------------------------------------------------
\label{cap2:sec:scripting}

Explicar la diferencia entre el desarrollo de un videojuego a trav�s de scripting por nodos y a trav�s de 
programaci�n. Hablar tambi�n sobre los sistemas de scripting investigados y explicar sus principales 
caracter�sticas.

%-------------------------------------------------------------------
\section{Librerias utilizadas}
%-------------------------------------------------------------------
\label{cap2:sec:librerias}

Librearias utilizadas y por que

%-------------------------------------------------------------------
\section{Mercado de los motores de videojuegos 2D}
%-------------------------------------------------------------------
\label{cap2:sec:mercado}

�Una secci�n para el mercado?

%-------------------------------------------------------------------
\section*{\NotasBibliograficas}
%-------------------------------------------------------------------
\TocNotasBibliograficas

Referencias a los motores 2D y sistemas de scripting investigados. 
Asi como documentacione y repositorios utilizados.



